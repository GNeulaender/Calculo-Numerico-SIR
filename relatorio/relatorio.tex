\documentclass[11pt, a4paper]{article}
\usepackage[margin=1in]{geometry}

\usepackage{graphicx}
\newcommand{\executeiffilenewer}[3]{
	\ifnum\pdfstrcmp{\pdffilemoddate{#1}}
	{\pdffilemoddate{#2}}>0
	{\immediate\write18{#3}}\fi
}
\newcommand{\includesvg}[1]{
	\executeiffilenewer{#1.svg}{#1.pdf}
	{inkscape -D #1.svg -o #1.pdf --export-latex}
	\input{#1.pdf_tex}
}

\usepackage[utf8]{inputenc}
\usepackage[portuguese]{babel}
\usepackage{csquotes}

\usepackage{hyperref}
\usepackage{biblatex}
\addbibresource{referencias.bib}

\usepackage{mathtools}
\usepackage{amssymb}
\usepackage{commath}
\newcommand{\R}{\mathbb R}

\begin{document}

\input{./title_page.tex}

\section{Introdução}
Esse projeto foi feito com intuito de compreender a evolução do coronavírus no Estado de Minas Gerais.
Nele foi utilizado o modelo compartimental SIR (Suscetíveis-Infectados-Removidos), que divide a população (N) em 3 grupos: suscetíveis a infecção (S); os que já foram infectados e que podem infectar os  (I); e os que já foram removidos (R), seja por terem sido curados ou pelo óbito. 
É sabido que esses 3 grupos interagem segundo o seguinte sistema de Equações Diferenciais Ordinárias (EDOs):
\[
	\od{S}{t} = - \gamma r_0 \frac{IS}{N} , \quad
	\od{I}{t} = \gamma \left( r_0 \frac{IS}{N} - I \right) \text{ e} \quad
	\od{R}{t} = \gamma I
\]
onde $r_0 = \frac{\beta}{\gamma}$ é constante ou dado em função do tempo. 
Para simulação desse sistema foi utilizado o método numérico de Runge-Kutta, que foi escolhido por ser um método bastante flexível, o que permite que sejam simulados  diversos cenários com diversas condições iniciais.
Vale ressaltar que foram utilizadas como base de dados as seguintes referências:
o painel coronavírus, organizado pelo Professor Alberto Saa \cite{git_saa};
o site da Wikipédia sobre \emph{Compartmental models in epidemiology} \cite{model}
e o livro Cálculo Numérico de Marcia A. G. Ruggiero e Vera L. R. Lopes \cite{calc_num}.
Ao final da pesquisa, conseguimos obter alguns resultados, disponibilizados em anexo ao final do relatório, que iremos discutir brevemente o que eles querem dizer e com toda implementação do código podendo ser encontrada no GitHub do grupo \cite{git_grupo}.

Para compreensão da evolução do coronavírus a partir do sistema de EDOs, dito anteriormente, foi-se avaliada duas possibilidades em relação ao $r_0$, a primeira com o seu valor constante e a segunda com seu valor variando em relação ao tempo.
Possivelmente, a principal diferença entre as duas possibilidades deve-se ao fato de que a segunda terá valores mais exatos do que a primeira, já que ela estará mudando sempre com o tempo devido a variação da taxa de infecção, enquanto a primeira terá sempre o mesmo valor independente das condições.

Para a primeira possibilidade, em que o $r_0$ é dito constante, adotamos seu valor como $r_0 \approx 2,6$, dado pela simples razão entre a taxa de infecção ($\beta \approx 0,34$) e a taxa de remoção ($\gamma \approx 0,13$).
Já a segunda possibilidade, que seria o $r_0$ em função do tempo, foi calculado a partir da seguinte fórmula, dados $\gamma$ e $\alpha$:
\[
	r_0(t) = \frac{1}{1 - \mu} + \frac{\ddot{C}}{\gamma \dot{C} (1 - \mu)}
	\text{, com} \quad
	\mu(t) = \frac{\alpha}{\gamma N} (\gamma C + \dot{C})
\]

\printbibliography

\end{document}

