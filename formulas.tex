\documentclass[12pt, a4paper]{minimal}
\usepackage[margin=1in]{geometry}
\usepackage{mathtools}
\usepackage{amssymb}
\usepackage{commath}
\newcommand{\R}{\mathbb R}

\begin{document}
Nós sabemos resolver pelo método de Runge-Kutta PVIs da forma
\[
	\left\{ \begin{array}{ll}
	y' &= f(x,y) \\
	y(x_{0}) &= y_{0}
	\end{array} \right.
\]
usando a formula de quarta ordem:
\[
	\begin{split}
	y_{n+1} &= y_n + \frac{1}{6} \left( k_1 + 2 k_2 + 2 k_3 + k_4 \right)
	, \quad \text{onde} \\
	k_1 &= \Delta x \cdot f(x_n, y_n) \\
	k_2 &= \Delta x \cdot f(x_n + \Delta x / 2, y_n + k_1 / 2) \\
	k_3 &= \Delta x \cdot f(x_n + \Delta x / 2, y_n + k_2 / 2) \\
	k_4 &= \Delta x \cdot f(x_n + \Delta x , y_n + k_3 )
	\end{split}
\]

Precisamos, então, reescrever as equações do modelo SIR
\[
	\od{S}{t} = - \gamma r_0 \frac{IS}{N} , \quad
	\od{I}{t} = \gamma \left( r_0 \frac{IS}{N} - I \right) \text{ e} \quad
	\od{R}{t} = \gamma I
\]
de forma que consigamos aplicar Runge-Kutta. Para isso, vamos vetorizar!
\[
	\vec{u} = \left( \begin{array}{rr}
	S \\
	I \\
	R
	\end{array} \right) \text{ e } \quad
	\vec{f}(t,\vec{u}) = \left( \begin{array}{rr}
	f_1 \\
	f_2 \\
	f_3
	\end{array} \right) \text{, com} \quad
	f_1 = - \gamma r_0 \frac{IS}{N} , \quad
	f_2 = \gamma \left( r_0 \frac{IS}{N} - I \right) \text{ e} \quad
	f_3 = \gamma I
\]
Dessa forma temos que
\[
	\od{\vec{u}}{t} = \vec{f}(t, \vec{u}) \text{ e} \quad
	\vec{u}(t_0) = u_0
\]
e portanto podemos aplicar Runge-Kutta:
\[
	\begin{split}
	\vec{u}_{n+1} &= \vec{u}_n + \frac{1}{6} \left( \vec{k}_1 + \vec{2 k}_2 + \vec{2 k}_3 + \vec{k}_4 \right)
	, \quad \text{onde} \\
	\vec{k}_1 &= \Delta t \cdot f(t_n, \vec{u}_n) \\
	\vec{k}_2 &= \Delta t \cdot f(t_n + \Delta t / 2, \vec{u}_n + k_1 / 2) \\
	\vec{k}_3 &= \Delta t \cdot f(t_n + \Delta t / 2, \vec{u}_n + k_2 / 2) \\
	\vec{k}_4 &= \Delta t \cdot f(t_n + \Delta t , \vec{u}_n + k_3 )
	\end{split}
\]

\end{document}

